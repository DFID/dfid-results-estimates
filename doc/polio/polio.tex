\chapter{Progress Towards Polio Eradication}

\section*{Number of global wild poliovirus cases.}

\bigskip
\bigskip

% make sure numbers and header don't appear
\thispagestyle{empty}

\section{Results}
DFID supports the Global Polio Eradication Initiative (GPEI) to eradicate polio, with \textbf{175 cases} of wild poliovirus (WPV) globally in 2019. \\%

\noindent\textbf{The number of wild polio cases has been reduced from 350,000 in 1988, to 175 in 2019}\footnote{\href{http://polioeradication.org/polio-today/polio-now/this-week/}{http://polioeradication.org/polio-today/polio-now/this-week/}}. \\%

In 2019, Afghanistan reported 29 cases and Pakistan reported 146 cases. %
Nigeria has not reported a case since August 2016 and discussions are ongoing about declaring the country, and the African Continent, free from WPV. \\%

An increase of 142 cases between 2018 (in which there were 33 cases) and 2019 was the result of severe insecurity and inaccessibility in certain parts of the endemic countries, as well as the spread of misinformation about the vaccine that prevented GPEI from performing vaccination campaigns and reduced vaccine uptake. %
The GPEI is working with a wide range of actors to improve access and build trust with local communities in the affected regions. %


\section{Context}

Polio is a highly infectious viral disease which primarily spreads from person to person, usually by the faecal-oral route. %
It causes irreversible paralysis in around 1 in 200 cases, most of whom are children under five years old. %
Among those paralysed, 5\% to 10\% will die. %
There is no cure for polio, but there are safe and effective vaccines. \\%

Since its inception in 1988, the GPEI, which is a partnership of World Health Organisation (WHO), United Nations Children's Fund (UNICEF), the Bill \& Melinda Gates Foundation, the US Center for Disease Control, Gavi the Vaccine Alliance \& Rotary International, has successfully led global efforts that have reduced Wild Polio Virus (WPV) cases by more than 99\% from 350,000 cases a year in 125 countries, to 175 cases, in 2019, with only three countries not yet certified wild polio-free (Afghanistan, Nigeria, Pakistan). \\%

The UK is currently the second largest state donor to the GPEI and is also the largest donor to Gavi, the Vaccine Alliance, which funds the Inactivated Polio Vaccine (IPV). %

\newpage
