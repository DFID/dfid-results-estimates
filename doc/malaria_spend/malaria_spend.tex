\chapter{Malaria: Spend}

\section*{Total UK government Official Development Assistance (ODA) spent on activities that contribute to prevention or treatment of malaria.} %

\thispagestyle{empty}

\section{Results}

In 2019/20 the total estimated UK government spend on malaria was \textbf{\pounds 429 million}. %

\section{Context}

Malaria is an infectious disease transmitted by mosquitos. %
Malaria caused an estimated 405,000 deaths in 2018 and children under the age of 5 are especially vulnerable to the disease. Most malaria cases in 2018 were in Africa (93\%) followed by South-East Asia (3.4\%)\footnote{World Malaria Report, 2018.}. %

Sustainable Development Goal 3, ``Ensure healthy lives and promote well-being for all at all age'', covers a range of health issues. %
Of the 13 targets, one (target 3.3) covers malaria specifically: ``By 2030, end the epidemics of AIDS, TB, malaria and NTDs and combat hepatitis, water-borne diseases and other communicable diseases.'' %
Progress on malaria is tracked by the SDG indicator ``Malaria incidence per 1,000 population.'' %

The UK is currently the second largest global funder of the effort against malaria\footnote{World Malaria Report, 2018.}. %
The UK contributes to the global effort on malaria through bilateral programming and funding to multilateral institutions including the Global Fund to Fight AIDS, Tuberculosis and Malaria and the World Health Organisation (WHO). %
The UK also funds research on the development of new drugs and diagnostics. %

In 2016, the then Chancellor of the Exchequer committed the UK to spend \pounds 500 million per year for five years (to 2020/21) on combating malaria. %
The then Prime Minister re-affirmed the \pounds 500 million commitment at the 2018 Commonwealth Summit. %
On 1\textsuperscript{st} July 2019, the UK government announced an up to \pounds 1.4 billion pledge to the Sixth Replenishment of the Global Fund to Fight AIDS, Tuberculosis and Malaria. %
This includes up to \pounds 200 million to double the value of private sector contributions to the Global Fund, providing \pounds 2 for every \pounds 1 contributed by the private sector. %

In 2016/17 the UK spend on malaria was \pounds 499 million, in 2017/18 the UK spend on malaria was \pounds 481 million, and in 2018/19 the UK spend on malaria was \pounds 452 million. %
These figures include all UK government funding on direct malaria programmes, multilateral contributions, research on the development of new drugs and diagnostics, and estimated contributions from wider programmes such as strengthening health systems in malaria affected countries. %


\newpage
