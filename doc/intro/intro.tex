\chapter*{Introduction}
\addcontentsline{toc}{chapter}{Introduction}

\thispagestyle{empty}

This report presents the latest update to our results estimates, with some background information about how and why we monitor each sector. %
Results estimates cover the period April 2015 to March 2020. %
The accompanying tables provide the same sectoral data, but without the background provided here. %
To fully understand the data we present in the tables and this report, the \href{https://assets.publishing.service.gov.uk/government/uploads/system/uploads/attachment_data/file/911810/draft-dfid_results-estimates_methodology-notes_2015-2020.pdf}{\textbf{Methodology Note}} for each indicator and the \href{https://assets.publishing.service.gov.uk/government/uploads/system/uploads/attachment_data/file/911809/dfid_results-estimates_technical-notes_2015-2020.pdf}{\textbf{Technical Note}} should be consulted, which contain further definitions, guidance and policies applied to the calculation of results estimates. %


\section*{What are results estimates?}
DFID collects data across its programmes to monitor its performance and to ensure that it is having a positive impact for the world's poorest. %
In 2015, DFID began using its Single Departmental Plan (SDP) as its main results framework, which consists of indicators covering a number of priority areas. %
Results estimates are figures that have been collated from a set of our programme results to provide an indication of our activity at an organisational level. %

\section*{Why are they only estimates?}
Programme data is collected from a wide variety of sources, by different partners, using different methods, operating in different contexts. %
As such, the quality of the data we receive varies and although we and our partners strive to apply data quality best practices, this can be particularly challenging in some of the environments in which we operate. %
This is one reason why we refer to our results as estimates and why they are not considered to be Official or National statistics. %
However, we do follow the \href{https://www.statisticsauthority.gov.uk/code-of-practice/}{Code of Practice for Statistics} in their production wherever possible.  This \href{https://assets.publishing.service.gov.uk/government/uploads/system/uploads/attachment_data/file/819426/DFID-Results-statement-voluntary-comp-Code-Practice-Statistics-2019.pdf}{statement on voluntary compliance} demonstrates the steps we have taken to improve the trustworthiness, quality and value of DFID's results estimates. %

The second reason we refer to our results (excluding spend indicators) as estimates is because the process for results aggregation is intentionally designed to estimate the \textbf{minimum} total reach of our aid spend in key areas. %
We must ensure that each beneficiary of our programmes is only counted once towards the total result for an indicator. %
We always try to avoid double-counting; preferring to take a conservative approach, rather than risk overstating our results estimates. %
More detail about the methods we use to estimate our results and technical information about data quality and presentation, can be found in our \href{https://assets.publishing.service.gov.uk/government/uploads/system/uploads/attachment_data/file/911810/draft-dfid_results-estimates_methodology-notes_2015-2020.pdf}{Methodology Notes} and \href{https://assets.publishing.service.gov.uk/government/uploads/system/uploads/attachment_data/file/911809/dfid_results-estimates_technical-notes_2015-2020.pdf}{Technical Notes}. %

\section*{Cumulative annual estimates}
In most cases we report our headline results estimates cumulatively for the whole spending review period, which means that the latest published results cover the total period from 2015 to 2020. %
We do not publish results for each individual year within that period. %
We do this because each year we may receive updated values from partners for previous years, as well as for the current year. %
In some cases, to avoid double counting, we need to use measures based on the `peak year' or `average' value which cannot be broken down by single years. %
Further information on this is provided in our \href{https://assets.publishing.service.gov.uk/government/uploads/system/uploads/attachment_data/file/911809/dfid_results-estimates_technical-notes_2015-2020.pdf}{Technical Notes}.

\section*{Delay due to COVID-19}
The global disruption caused by COVID-19, and the reprioritisation of DFID's efforts to help tackle the crisis in developing countries, impacted the capacity of our country offices and central staff to adequately quality assure our latest results data for 2019-20 in time for publication in Spring. %
Therefore, considering the Office for Statistics Regulation's guidance on statistical practice during the coronavirus outbreak, and our voluntary commitment to the Code of Practice for Statistics, the Chief Statistician decided to delay publication of updates for some results estimates until August 2020. %
This publication now comprises results for all of indicators including those which were delayed. %


\newpage
